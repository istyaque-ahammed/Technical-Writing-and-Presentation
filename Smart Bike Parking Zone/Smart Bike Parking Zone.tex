\documentclass[conference]{IEEEtran}
\IEEEoverridecommandlockouts
% The preceding line is only needed to identify funding in the first footnote. If that is unneeded, please comment it out.
\usepackage{cite}
\usepackage{amsmath,amssymb,amsfonts}
\usepackage{algorithmic}
\usepackage{graphicx}
\usepackage{textcomp}
\usepackage{xcolor}
\def\BibTeX{{\rm B\kern-.05em{\sc i\kern-.025em b}\kern-.08em
		T\kern-.1667em\lower.7ex\hbox{E}\kern-.125emX}}
\begin{document}
	
	\title{Arduino Uno Based Smart Bike Parking Zone\\
		

	}
	
	\author{\IEEEauthorblockN{Istyaque Ahammed \\ Student Id: 200240} 
		\IEEEauthorblockA{\textit{Computer Science and Engineering} \\
			\textit{Khulna University}\\
			Khulna, Bangladesh \\
			istyaque2040@cseku.ac.bd}
		\and
		\IEEEauthorblockN{Sumaiah Binta Musa \\ Student Id: 190205} 
		\IEEEauthorblockA{\textit{Computer Science and Engineering} \\
			\textit{Khulna University}\\
			Khulna, Bangladesh \\
			sumaiah1905@cseku.ac.bd}
	}
	
	\maketitle
	
	\begin{abstract}
		Bike parking in busy areas can be a challenge due to limited parking spots and the difficulties in detecting the presence of bikes. This study proposes a smart bike parking zone using Arduino Uno, a microcontroller-based system that aims to improve the efficiency and convenience of bike parking. The system utilizes ultrasonic sensors to detect the presence of bikes in parking spots and provide real-time information on parking availability. The system also includes a user interface that allows bike riders to locate available parking spots and an interface for parking administrators to manage the parking spots remotely. The results of this study show that the smart bike parking zone using Arduino Uno is an effective solution for managing bike parking in busy areas. The system provides accurate and real-time information on parking availability and has the potential to reduce the time and effort required for bike riders to find a parking spot. The findings of this study can provide valuable insights for future improvements in smart bike parking systems.
	\end{abstract}
	
	\begin{IEEEkeywords}
		Arduino Uno, bike parking, infrared sensors, smart parking zone, real-time information, user interface, parking availability, data analysis, IoT technologies.
	\end{IEEEkeywords}
	
	\section{Introduction}
	The Arduino Uno based smart bike parking zone is a project that aims to create a more efficient and convenient system for parking bikes. The project utilizes the Arduino Uno microcontroller to control the movement of bikes in and out of the parking lot, as well as to monitor the availability of parking spaces. 
	
	\section{Related Work}
	In recent years, there has been a growing need for efficient and smart parking solutions, especially in densely populated cities. The issue of finding a parking spot is a major concern for bike riders and can often lead to traffic congestion, parking violations, and even accidents. To address this issue, several studies have been conducted to develop smart parking systems that can efficiently manage the parking spaces.
	
	One of the most common approaches for smart parking systems is the use of sensor-based technologies. In this approach, sensors are placed in the parking zones to detect the presence of vehicles and transmit the data to a central control system. For example,in [1], an IoT-based smart parking system was proposed that used ultrasonic sensors to detect the presence of vehicles and RFID technology to track the parking spots. Similarly, in [2], a smart parking system was developed using infrared sensors and a GSM module to send real-time parking information to users.
	
	Another approach for smart parking systems is the use of computer vision techniques. In this approach, cameras are installed in the parking zones to detect the presence of vehicles and analyze the parking space utilization. For example, in [3], a computer vision-based parking management system was proposed that used a deep learning algorithm to classify the parking spots and detect vehicles in real-time.
	
	While these studies have made significant contributions to the development of smart parking systems, they often require expensive hardware and technical expertise to implement. In this study, we aim to develop a smart bike parking zone using an Arduino Uno microcontroller and a set of simple sensors. Our approach is designed to be cost-effective and easy to implement, making it accessible for a wide range of users.
	
	In conclusion, the development of smart bike parking zones is an important step towards improving the efficiency of parking and reducing the challenges faced by bike riders. Our proposed solution is based on the use of a simple and affordable microcontroller and sensor setup, making it a cost-effective and accessible solution for a wide range of users.
\begin{thebibliography}{00}
\bibitem{b11}Ismail, Mohd , Jusoh, Muzammil , Sabapathy, Thennarasan , Osman, M.N. , A Rahim, Hasliza , Yasin, Najib , Mohd Fazilah, Ainur. (2019). IoT Based Smart Parking System. Journal of Physics: Conference Series. 1424. 012021. 10.1088/1742-6596/1424/1/012021. 
\bibitem{b2} Üstün Ercan, Seda. (2020). Safety Based Smart Parking Guidance System Using GSM and Zigbee. 
\bibitem{b3} Almeida, Paulo , Alves, Jeovane , Parpinelli, Rafael , Barddal, Jean Paul. (2022). A systematic review on computer vision-based parking lot management applied on public datasets. Expert Systems with Applications. 198. 116731. 10.1016/j.eswa.2022.116731.
\end{thebibliography}

\end{document}
